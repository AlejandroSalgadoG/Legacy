\documentclass{article}

\usepackage[utf8]{inputenc}
\usepackage[spanish]{babel}
\usepackage{babelbib}
\usepackage{geometry}
\usepackage{wrapfig}
\usepackage{mathtools}

\geometry{letterpaper,tmargin=2cm,bmargin=2cm,lmargin=2cm,rmargin=2cm}

\begin{document}

\begin{titlepage}
    \centering
    \vspace{2cm}

    {\huge\bfseries Análisis de un sistema no lineal para el tratamiento del VIH\par}
    \vspace{5cm}
    {\Large\itshape Alejandro Salgado Gómez\par}
    \vfill

    Profesor\par
    {\large Santiago Lopez Restrepo\par}

    \vfill

    {\large Modelación y simulación IV \par}
    \vspace{0.2cm}
    {\large Ingeniería Matemática \par}
    \vspace{0.2cm}
    {\large Departamento de ciencias matemáticas \par}
    \vspace{0.2cm}
    {\large Escuela de ciencias \par}
    \vspace{0.2cm}
    {\large Universidad EAFIT \par}

    \vfill

    {\large Noviembre 22 de 2017}
\end{titlepage}


\section{Planteamiento del problema}

Las estrategias para contrarrestar el VIH utilizando métodos de control están
recibiendo cada vez más atención. Estudios detallados que combinan
tecnícas de modelado con resultados clínicos muestran que la fase inicial de la
infección puede ser representada utilizando modelos no lineales simples.\cite{paper}
Este hecho impulso la producción de artículos donde se trabaja con los principios
de control con el fin de estudiar diversas estrategias para implementar terapias de dicha
enfermedad.\\

En \cite{model} se encontró un modelo no lineal con tres variables de estado, en el
cual la dinámica de una de sus variables puede ser calculada por medio de una
ecuación algebráica al asumir su cambio como instántaneo en reacción a las variaciones
de otra variable, reduciendo así el modelo a uno de dos estados.

Este modelo asume la complejidad no lineal para representar de una manera 
mas acertada el efecto que las drogas usadas en el tratamiento de esta
enfermedad tienen en los pacientes, con el fin de buscar alternativas para
implementar estrategias de control en las terapias del VIH.\\

El problema que se desea abordar en este trabajo es el análisis en distintos niveles
de dicho modelo, con el objetivo de entender más a fondo su estructura y
funcionamiento, desde los detalles más generales hasta los más específicos.

\section{Objetivo general}

Analizar el sistema no lineal de ecuaciones diferenciales escogido, sus componentes y
como reacciona al cambiar sus condiciones con el propósito de aplicar la teoría
estudiada en clase.

\section{Objetivos específicos}

\begin{itemize}
    \item Reconocer parámetros, entradas y salidas del sistema escogido.
    \item Crear el respectivo diagrama de estados.
    \item Simular el sistema.
    \item Estudiar el acoplamiento de ecuaciones.
    \item Realizar un análisis completo de las gráficas.
    \item Análisis del sistema al variar entradas y condiciones iniciales.
    \item Hallar los puntos de equilibrio del sistema.
    \item Linealizar el sistema y hacer un análisis de estabilidad en los puntos de equilibrio.
    \item Implementar un sistema de control
\end{itemize}

\section{Antecedentes del modelo escogido}

Existen varios antecentes directamente relacionados con el trabajo realizado en el
artículo, como lo son \cite{ieee1}, \cite{ieee2}, \cite{ieee3}. Sin embargo
debido a la dificultad de conseguir dichos trabajos, se proponen como
antecedentes los siguientes modelos generales.

\newpage

    \subsection{Modelo SIR}

        \Large
        $$\dot{s} = -b s(t) i(t)$$
        $$\dot{i} = b s(t) i(t) - k i(t)$$
        $$\dot{r} = k i(t)$$
        \normalsize

        \vspace{0.5cm}

        \begin{tabular}[t]{|p{4cm} p{3.5cm}|}
            \hline
            \textbf{Variables de estado} & \textbf{Unidades} \\
            \hline
            $s$: Población suceptible & Número de individuos\\
            $i$: Población infectada  & Número de individuos\\
            $r$: Población recuperada & Número de individuos\\
            \hline
        \end{tabular}
        \hspace{0.5cm}
        \begin{tabular}[t]{|p{4cm} p{4cm}|}
            \hline
            \textbf{Parámetros} & \textbf{Unidades} \\
            \hline
            $b$: tasa de infección    & $1/(individuos * tiempo)$\\
            $k$: tasa de recuperación & $1/tiempo$\\
            \hline
        \end{tabular}
        \cite{sir}

        \vspace{0.5cm}

        En este modelo se busca plasmar la dinámica de tres poblaciones de
        individuos, los suceptibles, quienes todavía no han sido contagiados
        por el tipo de infección a modelar, los infectados quienes tienen la
        infección y los recuperados, quienes ya son inmunes.\\

        En el modelo la primera ecuación muestra el cambio en la población de
        suceptibles debido al contagio que producen los infectados, la
        segunda muestra como la población de los infectados varia
        dependiendo de la cantidad de contagiados y el número de recuperados, y
        por último en la tercera se determina como es el proceso de cura
        de los individuos infectados.

    \subsection{Modelo predador-presa}

        \Large
        $$\dot{p} = \alpha p(t) - \beta p(t) d(t)$$
        $$\dot{d} = \lambda p(t) d(t) - \gamma d(t)$$
        \normalsize

        \vspace{0.5cm}

        \begin{tabular}[t]{|p{4cm} p{3.5cm}|}
            \hline
            \textbf{Variables de estado} & \textbf{Unidades} \\
            \hline
            $p$: Población presas       & Número de individuos\\
            $d$: Población depredadores & Número de individuos\\
            \hline
        \end{tabular}
        \hspace{0.5cm}
        \begin{tabular}[t]{|p{4.5cm} p{4cm}|}
            \hline
            \textbf{Parámetro} & \textbf{Unidades} \\
            \hline
            $\alpha$:  tasa nacimiento presas   & $1/tiempo$\\
            $\beta$:   tasa caza                & $1/(individuos * tiempo)$\\
            $\lambda$: tasa alimentación        & $1/(individuos * tiempo)$\\
            $\gamma$:  tasa muerte depredadores & $1/tiempo$\\
            \hline
        \end{tabular}
        \cite{predator}

        \vspace{0.5cm}

        En este modelo se tienen dos poblaciones, presas y depredadores, los
        cuales presentan un comportamiento de crecimiento exponencial en
        ausencia del otro (positivo para las presas y negativo para los
        predadores). En este modelo la primera ecuación muestra como el
        crecimiento exponencial de las presas es diesmado por la acción de los
        depredadores y la segunda ecuación define como la cantidad de presas
        removidas influye en el decrecimiento de la población de depredadores.

\newpage

\section{Modelo escogido}

    \Large
    $$\dot{x_1} = s -dx_1 - (1-u_1) \beta x_1 x_3$$
    $$\dot{x_2} = (1-u_1) \beta x_1 x_3 - \mu x_2$$
    $$\dot{x_3} = (1-u_2) k x_2 - c x_3$$
    \normalsize

\vspace{1cm}

    \begin{tabular}{|p{6cm} p{2.5cm}|}
        \hline
        \textbf{Variables de estado} & \textbf{Unidades} \\
        \hline
        $x_1$: Concentración de células saludables & $celulas / mm^3$\\
        $x_2$: Concentración de células infectadas & $celulas / mm^3$\\
        $x_3$: Concentración de virus libre        & $copias / mm^3$\\
        \hline
    \end{tabular}

    \vspace{0.5cm}

    \begin{tabular}{|p{8cm} p{2.5cm}|}
        \hline
        \textbf{Entradas} & \textbf{Unidades} \\
        \hline
        $u_1$: Eficiencia para prevenir la infección              & Adimensional \\
        $u_2$: Eficiencia para evitar la creacion de virus libres & Adimensional \\
        \hline
    \end{tabular}

    \vspace{0.5cm}

    \begin{tabular}{|p{7cm} p{2cm} c|}
        \hline
        \textbf{Parámetro} & \textbf{Valor} & \textbf{Unidades} \\
        \hline
        $d$: tasa de muerte natural de las células      & 0.02            & $s^{-1}$\\
        $k$: tasa de producción de virus libres         & 100             & $s^{-1}$\\
        $s$: tasa de producción de células sanas        & 10              & $mm^{-3} s^{-1}$\\
        $\beta$: tasa de infección                      & $2.4 * 10^{-5}$ & $mm^3 s^{-1}$\\
        $\mu$: tasa de muerte de las células infectadas & 0.24            & $s^{-1}$\\
        $c$: tasa de muerte de los virus libres         & 2.4             & $s^{-1}$\\
        \hline
    \end{tabular}

\vspace{1cm}

En este modelo se busca plasmar la dinámica del virus VIH con el uso de tres
variables de estado que representan las células saludables, las infectadas y la
concentración de células libres en el cuerpo.\\

En la primera ecuación se muestra la dinámica de las células sanas y como su
disminución se ve afectada tanto por su muerte natural como por la influencia
del virus, en la segunda se expresa el cambio en las células
contaminadas como el contagio de células sanas menos una tasa de muerte natural
de células infectas y por último en la tercera se plasma el proceso
por el cual las células contaminadas liberan mas virus en el sistema, los
cuales a su vez también tienen una tasa de muerte natural que los disminuye.\\

Uno de los aspectos a destacar del modelo es la similitud con los modelos
propuestos como antecedentes, específicamente en aspectos como la reducción de
las poblaciones debido a la infección, su contagio y la muerte de los virus,
tal como en el modelo SIR, y en la decaída de la poblaciones por una tasa
de muerte, donde se puede ver cierta similitud con el modelo
predador-presa. En segundo lugar es importante resaltar como las tasa de
infección de las células sanas y la tasa de creación de virus libres pueden
ser influenciadas con el efecto de drogas que afecten el desarrollo del
virus. Dicha influencia esta reprecentada por la variables de entrada, donde $u_1$
toma valores entre 0 y 1, reprecentando la eficiencia de drogas ITR
(Inhibidores de la Transcriptasa Inversa) que pueden reducir la propagacion del
virus. Y la variable $u_2$, que toma valores en el mismo intervalo y representa
la acción de drogas PI (Inhibidores de proteasa), las cuales previenen que las
células infectadas produscan mas virus libres, permitiendo así ejercer un
control sobre el desarrollo del virus en el cuerpo.\\

Es importante resaltar que  en la actualidad ninguna droga llega a una
eficiencia del 100\% por lo cual un valor cercano a 1 en estas variables
ocaciona que la simulación tome una condición cada vez más teorica.\cite{model}\\

    \subsection{Modelo reducido}

    \Large
    $$\dot{x_1} = s -dx_1 - (1-u) \frac{\beta k}{c} x_1 x_2$$
    $$\dot{x_2} = (1-u) \frac{\beta k}{c} x_1 x_2 - \mu x_2$$
    \normalsize

    \vspace{0.5cm}

    \begin{tabular}{|p{6cm} p{2.5cm}|}
        \hline
        \textbf{Variables de estado} & \textbf{Unidades} \\
        \hline
        $x_1$: Concentración de células saludables & $celulas / mm^3$\\
        $x_2$: Concentración de células infectadas & $celulas / mm^3$\\
        \hline
    \end{tabular}

    \vspace{0.5cm}

    \begin{tabular}{|p{8.5cm} p{2.5cm}|}
        \hline
        \textbf{Entradas} & \textbf{Unidades} \\
        \hline
        $u$: esta entrada es usada para simplificar las ecuaciones del modelo,
             tiene un valor de $u_1+u_2-u_1 u_2$ & Adimensional \\
        \hline
    \end{tabular}

    \vspace{0.5cm}

    \begin{tabular}{|p{7cm} p{2cm} c|}
        \hline
        \textbf{Parámetro} & \textbf{Valor} & \textbf{Unidades} \\
        \hline
        $d$: tasa de muerte natural de las células      & 0.02            & $s^{-1}$\\
        $k$: tasa de producción de virus libres         & 100             & $s^{-1}$\\
        $s$: tasa de producción de células sanas        & 10              & $mm^{-3} s^{-1}$\\
        $\beta$: tasa de infección                      & $2.4 * 10^{-5}$ & $mm^3 s^{-1}$\\
        $\mu$: tasa de muerte de las células infectadas & 0.24            & $s^{-1}$\\
        $c$: tasa de muerte de los virus libres         & 2.4             & $s^{-1}$\\
        \hline
    \end{tabular}

    \cite{model}

    \vspace{0.5cm}

Este modelo es bastante similar al anterior, tanto en estructura como en
aplicación, se puede ver basicamente como su simplifación. Los cambios
realizados son, en primer lugar, la eliminacion de $x_3$ como variable de
estado al encontrar que su cambio respecto a las variaciones de $x_2$ sucede de
manera casi instantanea, lo que permite prescindir de $x_3$ como variable de
estado al expresarla como una ecuación algebraica, reduciendo asi el modelo a
segundo orden.  Y en segundo lugar, la combinación de $u_1$ y $u_2$ para formar una
nueva entrada, con el objetivo simplificar las ecuaciones y facilitar el
entendimiento del modelo.

\bibliographystyle{unsrt}
\bibliography{Anteproyecto}

\end{document}
