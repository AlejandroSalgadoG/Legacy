\documentclass{article}

\usepackage[utf8]{inputenc}
\usepackage[spanish]{babel}
\usepackage[]{algorithm2e}
\usepackage{program}
\usepackage{listings}

\begin{document}

\title{Numerical}
\author{Alejandro Salgado Gómez}

\maketitle

\section{Estructura del proyecto}

A continuación se muestra una estructura general del repositorio que
contiene el código del proyecto.

\begin{itemize}
    \item \textbf{Error:} Valores agregados del capítulo de teoría
        de errores.

    \item \textbf{Equations:} Valores agregados del capítulo de
        solución de sistemas de ecuaciones no lineales.


    \item \textbf{Matrix:} Valores agregados del capítulo de solucion
        de sistemas de ecuaciones lineales.

    \item \textbf{Informs:} Fuentes de los archivos de texto del
        proyecto.

    \item \textbf{Proyect:} Código fuente del proyecto

\end{itemize}

\subsection{Modo de compilación y ejecución}

Ahora se tratará el tema que hace posible que los métodos sean
ejecutables, la compilación. La herramienta escogida para compilar
este proyecto es \textbf{make}, dentro de cada uno de los métodos
implementados, ya sea un ejemplo, valor agregado, documento de
texto o parte del proyecto, se encontrará el archivo
\textbf{Makefile} que contiene las instrucciones necesarias para la
compilación.\\

A manera de comentario, el compilador escogido es \textbf{mpic++},
el cual es simplemente una envoltura para llamar a \textbf{g++} con
parámetros sobre las direcciones de la librería openmpi.\\

Para crear el ejecutable solo se debe ejecutar el comando
\textbf{make} dentro de la carpeta que contiene el código a ser
compilado. Luego para ejecutar el programa se puede utilizar el
comando \textbf{make exe}, el cual ejecutará el binario con
parámetros por defecto que mostraran el comportamiento del programa
con un caso estandar.

Es importante aclarar que cada método numérico tiene tambien una
ayuda para ser ejecutado dentro de su código, para utilizarlo se
debe llamar al método solo con la bandera \textbf{-h}.

\subsection{Ejecución con MPI}

Esta sección tiene la intención de servir como una breve
explicación de la manera de ejecutar programas que usen la
librería MPI tanto en forma distribuída como utilizando una
sola máquina.

\begin{itemize}
    \item Ejecución básica
\end{itemize}

Luego de haber compilado el método deseado con su respectivo
\textbf{Makefile} el binario debe ser ejecutado de la
siguiente manera:\\

\begin{lstlisting}[frame=single]
mpirun -np <Numero de procesos> ./<Binario> <Parametros>

    <Numero de procesos> = Numero de veces que el programa
                           mpirun realiza la llamada fork.
                           Es recomendable crear tantos
                           procesos como procesadores se
                           tengan.

    <Binario> = Nombre del binario a ser ejecutado

    <Parametros> = Parametros que se desean pasar a el
                   programa.

\end{lstlisting}



\textbf{NOTA:} En el caso de utilizar la bandera
\textbf{-h} se recomienda ejecutarlo de manera convencional.

\begin{itemize}
    \item Ejecución distribuída
\end{itemize}

Este tipo de ejecución es muy similar al anterior, la única
diferencia es el uso de la bandera -machinefile, la cual
indica la ruta a un archivo que debe contener el nombre de
las máquinas (hostname) en las que se va a ejecutar en
conjunto con el número de procesadores en cada una de ellas.

Para ejecutar de manera distribuída se debe utilizar el
comando de la siguiente manera:

\newpage

\begin{lstlisting}[frame=single]
mpirun -machinefile <file> -np <Numero de procesos>
		./<Binario> <Parametros>

	<file> = Ruta indicando la localizacion del
			 archivo que contiene las
			 especificaciones de ejecucion

    <Numero de procesos> = Numero de veces que el programa
                           mpirun realiza la llamada fork.
                           Es recomendable crear tantos
                           procesos como procesadores se
                           tengan.

    <Binario> = Nombre del binario a ser ejecutado

    <Parametros> = Parametros que se desean pasar a el
                   programa.

\end{lstlisting}



El archivo de configuración es una lista de lineas cada una
especificando un procesador con el nombre de máquina deseado,
por ejemplo\\

\begin{lstlisting}[frame=single]
localhost
localhost
compute-0
compute-0
compute-1
\end{lstlisting}



\vspace{0.5cm}

En el ejemplo anterior se esta indicando que en la máquina
local se van a ejecutar 2 procesos, al igual que en compute-0
y solo uno en compute-1 para un total de 5 procesos que se
ejecutarán de manera distribuída.\\

\textbf{NOTA:} En caso de no conocer el nombre de la máquina
es posible reemplazarlo por su respectiva dirección IP, esto es
equivalente.

\subsection{Implementaciones básicas de paralelismo}

En esta sección se describirán los ejemplos que se utilizaron como
base para la creación del proyecto, entre ellos se encuentran
paralelización con llamadas al sistema, a nivel de hilos y a nivel
de proceso. Además se incluyeron también ejemplos para manejar
problemas de programación paralela como la sincronización y el uso
adecuado de variables compartidas.

\subsubsection{Llamadas al sistema}

En el sistema operativo linux existe sólo una manera de que un
un proceso pueda crear a otro para lograr asi el paralelismo,
esta llamada tiene por nombre \textbf{fork}. Es importante aclarar
que esta sección y sus ejemplos fueron implementados sólo a
manera de demostración, esto debido a la complejidad que
implicaría su uso en el proyecto, ya que sería necesario incluir
mas código y lógica de control en cada uno de los métodos
trabajados en el proyecto.\\

Los ejemplos implementados fueron:

\begin{itemize}
    \item \textbf{Basic:} Ejemplo básico para mostrar el
        funcionamiento de la llamada al sistema.

    \item \textbf{Execv:} Ejemplo que combina la llamada
        \textbf{fork} con \textbf{execv} para lograr crear
        no solo instancias del proceso actual sino de cualquier
        otro proceso.

    \item \textbf{File:} Ejemplo para demostrar el manejo de
        archivos entre los procesos creados con este método.

    \item \textbf{Var:} Ejemplo para mostrar como funciona el
        uso de variables después de utilizar la llamada
        al sistema.
\end{itemize}

\subsubsection{Hilos}

Otra de las formas más utilizadas para paralelizar un programa
es el uso de hilos de ejecución, este método tiene varias
ventajas entre ellas estan el hecho de tener por defecto un
espacio compartido de variables globales, su rapidez en la
creación y gran cantidad de documentación.\\

Por otro lado tiene ciertas desventajas como lo son su
complejidad en código, la continua aparición de problemas
concurrentes, en especial condiciones de carrera debido a su
espacio de direcciones compartido, hace complejo programar
correctamente y principalmente su gran dificultad para lograr
que el código sea ejecutado por varias máquinas al tiempo, lo
cual, desde el punto de vista de esta proyecto, supone un
problema importante ya que uno de sus principales objetivos es
la implementación de métodos que puedan ser ejecutados de
manera paralela.\\

Los ejemplos implementados fueron:

\begin{itemize}
    \item \textbf{Hello:} Ejemplo básico de utilización de hilos

    \item \textbf{Join:} Ejemplo básico de sincronización.

    \item \textbf{Semaphores:} Ejemplo para mostrar el correcto
        uso de variables compartidas para evitar problemas de
        concurrencia.

    \item \textbf{Mutex:} Ejemplo para mostrar como crear
        condiciones de exclusión mutua para evitar condiciones
        de carrera.

    \item \textbf{CondVar:} Ejemplo que muestra como es la
        utilización de variables condición para manejar señales
        y sincronización.
\end{itemize}

\subsubsection{OpenMp}

OpenMp es una librería que permite manejar los hilos de una
manera mucho menos compleja, su implementación esta basada en
pragmas lo cual hace que el código sea mucho mas entendible.\\

Sin embargo, sigue teniendo el mismo problema que los hilos,
una complejidad alta para lograr ser ejecutado de manera
distribuída, por esta razón esta sección, al igual que la de
llamadas al sistema fue implementada de modo ilustrativo.\\

Los ejemplos implementados fueron:

\begin{itemize}
    \item \textbf{NumThreads:} Ejemplo básico del uso de OpenMp.
    \item \textbf{For:} Paralelización de ciclos \textbf{for}
        convencionales.
\end{itemize}

\subsubsection{MPI}

La interfaz de paso de mensajes (Message passing interface),
es una de las librerías mas populares en el mundo de la
programación paralela debido a facilidad que provee en
comparación con otros métodos de paralelización como lo son las
llamadas al sistema. Además, como su nombre lo indica, implementa
la metodología de paso de mensajes y tener soporte para
programación distribuída, lo cual la hace la librería perfecta
para este proyecto.\\

Debido a complicaciones técnicas y de tiempo, no fue posible
explotar todo el potencial de esta librería en el proyecto,
sin embargo se deja como una tarea para el futuro. Debido a esta
razón todos los métodos son ejecutados de manera local, pero
en caso de continuar con el proyecto posteriormente, no sería
necesario cambiar el código debido a las facilidades que
proporciona la librería, solamente serían necesarias unas
cuantas modificaciones en el modo de ejecución. Para más
información ver la sección 2.3.1.\\

Los ejemplos implementados fueron:

\begin{itemize}
    \item \textbf{Initialization:} Ejemplo básico del uso de la
        librería.

    \item \textbf{ListProces:} Ejemplo para mostrar el uso de
        procesos en paralelo.

    \item \textbf{SendReceive:} Ejemplo para mostrar el uso del
        paso de mensajes.

    \item \textbf{Distributed:} Ejemplo para mostrar la forma
        de ejecución distribuída.

    \item \textbf{SendReceiveDistributed:} Ejemplo para mostrar
        la forma de combinar el paso de mensajes con la
        ejecución distribuída.
\end{itemize}

\subsection{Fórmula general de ecuaciones cuadráticas en paralelo}

\subsubsection{Descripción}

Esta sección muestra un programa que calcula en paralelo las dos soluciones
a una ecuación cuadrática dada. El proceso principal recibe el número de procesos
a usar y los identificadores de cada uno, luego recibe los coeficientes de la ecuación
y crea un proceso hijo al que le va a asignar el cálculo de la solución negativa de
la ecuación, así mismo, el proceso padre calculará la solución positiva. Por último, cada
proceso imprime su respectiva solución. A continuación se presenta el respectivo pseudocódigo
del programa, incluidos están el proceso principal y los procesos esclavos.

\subsubsection{Seudocódigo proceso principal}

\begin{program}
    rnk = Identificador\_de\_proceso
    rnk\_size = Numero\_de\_procesos\_totales

    | Leer | a \rcomment{Primer coeficiente}
    | Leer | b \rcomment{Segundo coeficiente}
    | Leer | c \rcomment{Tercer coeficiente}
\end{program}

\begin{program}
    \IF(rank == 0) \rcomment{Si es el proceso padre}
    sol = cuadratic\_pos(a, b, c) \rcomment{Encuentra la solución positiva}
    print(sol) \rcomment{Imprime la solución positiva}
    \ELSE \rcomment{Si es el proceso hijo}
    sol = cuadratic\_neg(a, b, c) \rcomment{Encuentra la solución negativa}
    print(sol) \rcomment{Imprime la solución negativa}
\end{program}

\subsubsection{Seudocódigo proceso esclavo (Positivo)}

\begin{program}
    internal = b^2-4*a*c \rcomment{Cálculo interno de la raíz}
    sol = (-b + \sqrt[]{internal}/(2*a) \rcomment{Cálculo de la solución positiva}
    return(sol)
\end{program}

\subsubsection{Seudocódigo proceso esclavo (Negativo)}

\begin{program}
    internal = b^2-4*a*c \rcomment{Cálculo interno de la raíz}
    sol = (-b - \sqrt[]{internal}/(2*a) \rcomment{Cálculo de la solución negativa}
    return(sol)
\end{program}

\subsection{Suma de Matrices en paralelo}

\subsubsection{Descripción}

Esta sección muestra un programa que suma dos matrices de forma paralela, la entrada del programa
va a recibir el número de procesos totales que se usarán para el cálculo, los identificadores
de dichos procesos y el tamaño de las matrices (N) que se van a sumar. El proceso padre primero crea e 
inicializa las matrices A y B de tamaño N con valores aleatorios, luego distribuye la matriz de forma
que sea equitativa la carga de trabajo para cada proceso. Luego para todos los procesos esclavos, se les
envía y asigna los valores de A y B que van a procesar, cada proceso devuelve una matriz resultado C. El proceso
padre vuelve a armar la matriz C con todas las submatrices calculadas por los procesos esclavos y la imprime.

\subsubsection{Seudocódigo proceso principal}

\begin{program}
    rnk = Identificador\_de\_proceso\\
    rnk\_size = Numero\_de\_procesos\_totales

    | Leer | n \rcomment{Tamaño de la matriz}

    \IF(rank == 0)\ \rcomment{Si es el proceso padre}

    A = nuevaMatriz(n) \rcomment{Crea una matriz vacía de n x n}
    B = nuevaMatriz(n) \rcomment{Crea una matriz vacía de n x n}
    initializeMatrix(A, B, n) \rcomment{Rellena las matrices con valores aleatorios}
\end{program}

\begin{program}
    distribution[] = distributeMatrix(n,rnk\_size)
\end{program}

\begin{program}
    \FOR (i=0 ; i < rnk\_size-1; i++)\{  \rcomment{Para todos los procesos esclavos}
    arr\_st = distribution[i]->start
    arr\_sz = distribution[i]->size

    |send|(A+arr\_st, i+1) \rcomment{Envia el valor de las variables}
    |send|(B+arr\_st, i+1) \rcomment{Al proceso con id = i+1}
    |send|(arr\_sz, i+1)

    C = nuevaMatriz(n) \rcomment{Se crea una matriz vacía de tamaño n x n}
    \}
\end{program}

\begin{program}
    \FOR (i=0 ; i < info[0].size; i++)
    C[i] = A[i] + B[i]
\end{program}

\begin{program}
    \FOR (i=0 ; i < rnk\_size-1; i++)\{  \rcomment{Para todos los procesos esclavos}
    arr\_st = distribution[i]->start
    arr\_sz = distribution[i]->size

    c = nuevoArreglo(arr\_sz)

    | receive |(c, arr\_sz) \rcomment{Recibe los datos de la nueva matriz}
    \}
\end{program}

\begin{program}
    \FOR (j=0 ; j< arr\_sz; j++)
    C[arr\_st++] = c[j];
\end{program}

\begin{program}
    print(C) \rcomment{Imprime la matriz resultado C}
\end{program}

\subsubsection{Seudocódigo procesos esclavos}

\begin{program}
    | receive |(arr\_size) \rcomment{Recibe los tamaños de las matrices}
    A = nuevaMatriz(arr\_size) \rcomment{Crea las submatrices A y B a procesar}
    B = nuevaMatriz(arr\_size)
    | receive |(A, arr\_size)  \rcomment{Recibe los valores de A y B para sumar}
    | receive |(B, arr\_size)
    C = nuevaMatriz(arr\_size) \rcomment{Crea la submatriz C para guardar el resultado}
\end{program}

\begin{program}
    \FOR (i = 0; i < arr\_size; i++) \rcomment{Calcula y guarda el resultado de la suma en C}
    C[i] = A[i] + B[i]
\end{program}

\begin{program}
    | send |(C, arr\_size) \rcomment{Envia el resultado}
\end{program}

\subsection{Manejo de matrices de gran dimensión}

\subsubsection{Descripción}

La intención que se tuvo con esta sección es implementar estrategias de tratamiendo de matrices
que debido a su tamaño, no se pueden tratar en su totalidad puesto que no hay suficiente espacio
en memoria para ellas. Se introduce entonces el tratamiento por lotes, en el que dividimos la matriz
en tal forma que podamos trabajarla por partes más pequeñas. La idea es lograr generar submatrices de tamaño N
a partir de una matriz A de gran dimensión. Por motivos de tiempo, no se pudo implementar el procesamiento
de dicha matriz a través de la división debido a que los algoritmos hay que reimplementarlos de forma
que procesen todas las matrices como un todo y no por aparte. A continuación se presentará el pseudocódigo que dará cuenta de
la creación y la división de las submatrices a partir de la matriz de gran dimensión A.

\subsubsection{Seudocódigo}

\begin{program}
| Leer | lotSize \rcomment{Tamaño de las submatrices}
| Leer | bSize \rcomment{Tamaño de la matriz grande}

contSubH = 0 \rcomment{Contador de las submatrices hacia la derecha}
contSubV = 0 \rcomment{Contador de las submatrices hacia abajo}

file = path \rcomment{Ruta a guardar las sub matrices}

\WHILE(contSubV < bSize)\{ \rcomment{Se leen los valores de cada }
i = 0 \rcomment{lote partiendo la matriz grande en lotSize }
\WHILE(contSubH < bSize)\{ 
values[lotSize][lotSize]; 
j = 0
\WHILE(i < lotSize)\{
    \WHILE(j < lotSize)\{
        values[i][j] = read(file, lotSize, i, j)
j++
\}
i++
\}
contSubH = contSubH + lotSize \rcomment{Se leen bloques de }
\}    \rcomment{izquierda a derecha}        
contSubV = contSubV + lotSize \ \rcomment{Y luego de arriba}
\} \rcomment{ a abajo}
\end{program}



\begin{program}

    nuevo\_archivo-i-j = write(new\_path, i, j, values)

\end{program}

\subsection{Jacobi paralelo}

\subsubsection{Seudocódigo proceso principal}

\begin{program}
    rnk = Identificador\_de\_proceso\\
    rnk\_size = Numero\_de\_procesos\_totales
    | Leer | n \rcomment{Tamaño de la matriz}
    | Leer | toler \rcomment{Tolerancia del método}
    | Leer | niter \rcomment{Número de iteraciónes}
    Ab = |readMatrix|(n) \rcomment{Leer matriz}
    x = |readVector|(n) \rcomment{Leer vector inicial}
\end{program}

\begin{program}
    ditribution[] = |distributeMatrix|( n, rnk\_size ) \rcomment{Devuelve la posición inicial}
    \rcomment{Y el número de columnas}
    \rcomment{Asignadas a cada proceso}
\end{program}

\begin{program}
    \FOR (i=0 ; i < rnk\_size-1; i++)\{  \rcomment{Para todos los procesos esclavos}
    row\_st = distribution[i]->start
    row\_cnt = distribution[i]->size

    |send|(n, i+1) \rcomment{Envia el valor de las variables}
    |send|(row\_st, i+1) \rcomment{Al proceso con id = i+1}
    |send|(row\_cnt, i+1)
\end{program}

\begin{program}
    Ab\_temp = |getSubMatrix|(Ab, distribution[i]); \rcomment{Obtener la submatriz}
    |send|(Ab\_temp, i+1)                           \rcomment{Correspondiente con}
    \rcomment{el proceso i+1 y enviarla}
    \}
\end{program}

\begin{program}
    error = toler + 1 \rcomment{Inicializar el error}
    |print|(x)  \rcomment{Imprimir el vector inicial}
    x\_n[n]  \rcomment{Espacio reservado para el nuevo vector}
\end{program}

\begin{program}
    seguir\_ejecucion = true \rcomment{Indicador de ejecución en esclavos}
    while(error > toler \&\& cnt < niter)\{
\end{program}

\begin{program}
    \FOR (i=1 ; i < rnk\_size-1; i++)\{  \rcomment{Para todos los procesos esclavos}
    |send|(keep\_going, i); \rcomment{Enviar indicador de continuar ejecución}
    |send|(x, i); \rcomment{Enviar vector actual}
    \}
\end{program}

\begin{program}
    \FOR (i=1 ; i < rnk\_size-1; i++)  \rcomment{Para todos los procesos esclavos}
    |receive|(x\_n, i); \rcomment{Recibir nuevo vector}
\end{program}

\begin{program}
    error = |maxNorm|(x,x\_n) \rcomment{Calcular el error con la norma máximo}
    |printInfo|(niter, x\_n, error) \rcomment{Imprimir el estado del método}
    x = x\_n \rcomment{Actualizar el vector}
    cnt = cnt + 1
    \}
\end{program}

\begin{program}
    keep\_going = 0
    \FOR (i=1 ; i < rnk\_size-1; i++)  \rcomment{Para todos los procesos esclavos}
    |send|(keep\_going, i); \rcomment{Enviar indicador de parar ejecución}
\end{program}

\begin{program}
    \IF (error < toler)
    \COMMENT{Raíz encontrada con tolertancia = toler}
    \ELSE
    \COMMENT{No se pudo encontrar una raíz con niter iteraciónes}
\end{program}


\subsubsection{Seudocódigo procesos esclavos}

\begin{program}
    row\_sz  \rcomment{Variables para manejar}
    row\_st  \rcomment{correctamente la}
    row\_cnt \rcomment{submatriz}
    |receive|(row\_sz , 0)
    |receive|(row\_st , 0) \rcomment{Recibir dichas variables}
    |receive|(row\_cnt, 0)
\end{program}

\begin{program}
    arr\_sz = (row\_sz+1) * row\_cnt \rcomment{Calcular el tamaño de la submatriz}
    Ab[arr\_sz] \rcomment{Reservar el tamaño de la submatriz}
    x[row\_sz] \rcomment{Reservar el tamaño de los vectores}
    x\_n[row\_cnt]
    |receive|(Ab, 0) \rcomment{Recibir la submatriz del proceso maestro}
\end{program}

\begin{program}
    |receive|(seguir\_ejecucion, 0) \rcomment{Recibir indicador de ejecución del proceso maestro}
\end{program}

\begin{program}
    calc\_acum \rcomment{Acumulador de cálculos}
    \WHILE (seguir\_ejecucion)\{
\end{program}

\begin{program}
    |receive|(x, 0) \rcomment{Recibir vector del proceso maestro}
\end{program}

\begin{program}
    \FOR ( i=0; i<row\_cnt; i++)\{  \rcomment{Para cada fila}
    calc\_acum = 0
\end{program}

\begin{program}
    \FOR ( j=0; j<row\_cnt; j++)  \rcomment{Calcular acumulado}
    \IF ( j == i+row\_st) \COMMENT{Ignorar iteración}
    \ELSE calc\_acum = calc\_acum + Ab[i][j] * x[j]
\end{program}

\begin{program}
    x\_n[i] = ( Ab[i][row\_sz] - calc\_acum) / Ab[i][i+row\_st] \rcomment{Calcular nuevo}
    \rcomment{Valor del vector}
    \}
\end{program}

\begin{program}
    |send|(x\_n,0) \rcomment{Enviar nuevo vector al proceso maestro}
\end{program}

\begin{program}
    |receive|(seguir\_ejecucion, 0) \rcomment{Recibir indicador de ejecucion}
    \rcomment{del proceso maestro}
    \}
\end{program}

\subsection{Valores agregados}

En esta sección se enumeran los programas que fueron creados como
valor agregado, para más información ver la sección 4.

\subsubsection{Implementaciones referentes a la teoría de error}

\begin{itemize}
    \item Epsilon de la máquina
    \item Números de punto flotante
    \item Menor número positivo
    \item Series de taylor

        \begin{itemize}
            \item Coseno
            \item Seno
            \item Exponencial
            \item Ln
        \end{itemize}
\end{itemize}

\subsubsection{Implementaciones referentes a ecuaciones no lineales}

\begin{itemize}
    \item Búsqueda incremental
    \item Bisección
    \item Punto fijo
    \item Regla falsa
    \item Newton
    \item Secante
    \item Raíces múltiples
\end{itemize}

\subsubsection{Implementaciones referentes a sistemas de equaciones
lineales}

\begin{itemize}
    \item Jacobi
\end{itemize}

\section{Información adicional}

Información adicional sobre el proyecto, seudocódigos no incluídos en
este documento y códigos de todos los programas realizados durante el
proyecto pueden ser encontrados en el repositorio de github\\
\textbf{https://github.com/AlejandroSalgadoG/Miscellaneous/Numerical}.

\end{document}

